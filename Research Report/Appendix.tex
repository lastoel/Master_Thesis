The first condition is the presence of a well-defined causal question. A question is well-defined when all relevant stakeholders of the ILSA agree on the priority of obtaining an answer to the question, which is articulated through the entire ILSA framework.\newline

\textit{Condition 1: the causal question must be well-defined and stem from a theoretical framework that is presumably of interest to governing bodies responsible for policy priorities.}  \newline

The second condition Kaplan specifies is that the question should be framed as a counterfactual question that is capable of yielding a real-life manipulation or intervention. Specifically in the case of ILSAs, the form of the question must have cross-cultural comparability. To properly define a counterfactual question, one must define the context in which the causation takes place. However, to meet the first assumption of Rubin's causal model, strong ignorability of treatment assignment, we would theoretically have to account for every covariate that could possibly be of influence on the treatment assignment. Especially when trying to answer causal questions on an international scale, this is virtually impossible, since infinitely many covariates can differ on the country level that may affect treatment assignment. To help select which covariates are relevant to the treatment and the effect, Kaplan puts forward Mackie's theory on causation. \newline

Mackie states that a factor that can be identified as the cause of an effect under some conditions, might not be the cause of that same effect under different conditions. He suggests that the issue of distinguishing between causes and conditions is addressed by properly defining the \emph{causal field} in which the causal relationship takes place \parencite{mackie_cement_1974}. In Mackie's theory, one defines a causal field by isolating the set of \emph{conjunctions}. Conjunctions are sets of factors, where each set sufficient but not necessary to the effect, and the whole set of conjunctions form a condition that is both necessary and sufficient to the effect. Our interest lies with the properties of the individual factors that make up these conjunctions, since those factors are the variables we can measure. When a conjunction is sufficient, but not necessary to an effect and each individual factor in that conjunction is not sufficient to the effect itself, but without that factor, the conjunction is not sufficient to the effect anymore, we speak of a factor with the INUS property: an \emph{insufficient} but \emph{non-redundant} part of an \emph{unnecessary} but \emph{sufficient} condition. To satisfy this second condition, we must identify the factors that serve as an INUS condition to the effect of interest. \newline

\textit{Condition 2: the causal question is framed as a counterfactual question, capable of yielding a real-life manipulation or intervention within the framework of a randomized experiment} \newline

While Mackie's notion of a causal field helps narrow down the number of covariates that need to be collected, the satisfaction of the strong ignorability assumption still depends on the availability of those covariates. This is captured in the third condition. The agency to determine which questions are incorporated in the questionnaire of course lies with the content experts designing the ILSA questionnaires. \newline

\textit{Condition 3: the collection of ancillary covariate information is relevant to the causal question of interest} \newline

The last condition pertains to the choice of statistical model and the robustness of the resulting estimand. Naturally, the statistical model used for the analysis must be chosen such that it actually yields an estimand that is of interest to the question at hand. More intricately, the statistical model must allow us to evaluate the obtained estimand against violations of causal assumptions. It is, for example, quite plausible that not all relevant covariates are captured in the questionnaire, violating conditions 2 and 3. To test against such violations, Kaplan suggests performing a series of sensitivity analyses by incorporating several reasonable values on possible unobserved confounders and measuring their effect on the estimand, thus controlling for hidden biases. \newline

\textit{Condition 4: the choice of statistical model provides the appropriate causal estimand accounting for the ancillary covariate information and allows that estimand to be tested in a sequence of sensitivity analyses} \newline