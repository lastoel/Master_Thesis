\documentclass{article}
\usepackage{setspace}
\onehalfspacing
\usepackage[utf8]{inputenc}
\usepackage{enumerate}

\usepackage[backend=biber, style=apa, sorting=nty]{biblatex}
\addbibresource{thesisbibliography.bib}

\begin{document}

\begin{titlepage}
   \begin{center}
       \vspace*{1cm}
       
       \huge
       \textbf{Comparative Causal Inference within PISA}
       
       \Large
       \vspace{0.5cm}
        RESEARCH REPORT
            
       \vspace{1.5cm}

       \textbf{Lauke Stoel (6899544)}\\
       
       \vspace{0.5cm}
       Supervisors: Marieke van Onna (CITO) and \newline
       Remco Feskens (CITO and University of Twente)

       \vfill
       
       \large     
       \textit{Methodology and Statistics for the Behavioural, Biomedical and Social Sciences}\\
       
       \vspace{0.5cm}
       \textit{Utrecht University}\\
            
       \vspace{0.8cm}
       
       Date: 19/12/2021 \\
       Word count: 2463 \\
       
       \vspace{0.5cm}
       FETC-approved: 21-1939\\
       
       \vfill
       
       \textit{Candidate journal: Large Scale Assessments in Education}
       
       
            
   \end{center}
\end{titlepage}


\newpage
\section{Introduction}
\subsection{Background and research question}

Myriads of valuable data on pupil's cognitive ability are periodically collected through International Large-Scale Assessments (ILSA's) such as the Programme for International Student Assessment (PISA). PISA is a triennial survey of 15-year-old students around the world that assesses their knowledge and skills in three core domains: reading, mathematics and science. Through PISA, the Organisation for Economic Co-operation and Development (OECD) aims to provide internationally comparable evidence on student performance \parencite{oecd_pisa_2020}. Ideally, these data could be used to compare different educational systems, to evaluate what features of each system lead to more favourable educational outcomes and inform national policy. However, it is impossible to conduct a randomised controlled trial and due to the observational nature the data, most methods used to analyse them are inept to distinguish descriptive quantities from causal effects \parencite{rubin_potential_2004}.

%What would theoretically be needed to ascribe differences in educational outcomes to differences in national policy, is an experiment in which different countries with effectively the same student population are each randomly assigned a treatmentment condition. Needless to say, such an experiment would be both unethical and impossible to conduct. 

Rubin's potential outcome model provides a framework in which such causal questions could theoretically be addressed \parencite{rubin_potential_2004}. It entails a theoretical experiment where we treat the circumstance of interest one person is subject to as treatment condition A, and theorise that this person was simultaneously assigned treatment condition B, of which the data are non-existing. Those data are treated as missing and are estimated based on existing data from people with similar background characteristics, who are subject to treatment condition B. In a recent review of research on causal inference with large-scale assessments in education, Kaplan (2016) proposes an approach to causal inference within Rubin's potential outcomes framework that could theoretically expose causal links between educational outcomes and features of educational systems. However, his approach has a disclaimer: it has not yet been tested on existing data and currently available software is likely insufficient to fully execute the proposed statistical models \parencite{kaplan_causal_2016}.

The present thesis tests Kaplan's approach to causal inference on a currently relevant question within the Netherlands, namely whether it would aid the equality of educational opportunity (educational outcome) if the age at which pupils are selected into educational tracks is delayed (feature of an educational system). This topic is much discussed, as previous research suggests more heavily tracked educational systems are associated with a bigger negative influence of SES on the equality of educational opportunity \parencite{veldhuis_onderwijscrisis_2021, bol_curricular_2014}.

Kaplan's approach to causal inference could provide a more comprehensive way of addressing such questions, but is expected to have practical limitations. Another, theoretical, limitation is that Kaplan's approach is designed treat the ability level of a pupil as the outcome variable, whereas the above question requires an analysis with a relationship as the outcome variable, namely the one between SES and assessment outcomes. %there is not an effect of SES on score for each pupil. 
Therefore, the research question the current project addresses is: \textit{to what extent can Kaplan's approach to causal inference be applied to inform educational policy regarding tracking, within the confines of currently available software?}

\subsection{Analytical Plan} 
\hspace{\parindent} \textbf{Data} \\
We use the data from PISA 2018, which are publicly available. The data set contains the responses of 10,215 pupils from 79 participating countries and background information on the pupils, their parents and schools. We enrich this data set with relevant country-level characteristics, such as the age at which pupils are selected into tracks. We perform all analyses in RStudio \parencite{RStudio}. 

\subsubsection*{\underline{Applying Kaplan's approach:}}

\hspace{\parindent} \textbf{Phase 1} - \textit{Bayesian propensity score estimation (BPSE)} - risk: medium \newline 
Select relevant background characteristics to construct a propensity score equation for each pupil with, using R package "MatchIt" \parencite{MatchIt}. For the initial analysis we use data from just two countries: the Netherlands and one other country, subject to educational tracking at a later age. Result: a subsample of pupils from both countries that is comparable on the selected background characteristics.\\

\textbf{Phase 2} - \textit{Estimating the causal effect} - risk: low \newline
Identify the average treatment effect of being subject to tracking from a later age onward on ability estimates. Construct the Central Credible Interval around the estimate, indicating whether the effect is of significant influence.\\

\textbf{Phase 3} - \textit{Test against violations of assumptions} - risk: high \newline
Perform sensitivity analysis on the treatment effect by including plausible hidden biases into the BPSE.

\subsubsection*{\underline{Extending Kaplan's approach:}} 
\hspace{\parindent} \textbf{Phase 4} - \textit{Exploring limitations} - risk: high
\begin{enumerate}[a)]
    \item Include multiple countries, exposing the challenge of incorporating multiple treatment conditions or treating tracking as a continuous variable.
    \item Estimate the effect of tracking on the relationship between SES and educational outcome. 
\end{enumerate}

\noindent The end result will be a comprehensive overview of the strengths and limitations of Kaplan's approach as applied to real large-scale assessment data and its value in informing Dutch educational policy.  

\newpage
\printbibliography
\nocite{*}
\end{document}


